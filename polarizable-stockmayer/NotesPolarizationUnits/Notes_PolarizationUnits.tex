\documentclass[aps, 12pt, amsmath, amssymb, onecolumn, notitlepage, nofootinbib]{revtex4-1}

\usepackage{graphicx}% Include figure files
\usepackage{dcolumn}% Align table columns on decimal point
\usepackage{bm}% bold math
\usepackage{amsmath} 
\usepackage{epstopdf}
\usepackage{epsfig}
\usepackage{subfigure}
\usepackage{fancyvrb}
\usepackage{tikz}      % for drawing graphics
%\usepackage[section]{placeins}

\renewcommand{\baselinestretch}{1.20}\normalsize

\newcommand{\br}{\mathbf{r}}
\newcommand{\bk}{\mathbf{k}}
\newcommand{\dd}[2]{\frac{\partial #1}{\partial #2}}

\allowdisplaybreaks

\begin{document}
%%%%%%%%
% title
%%%%%%%%

\title{{\small\it{Notes}}\\ Notes on the Units for Polarizability }
\author{Bilin Zhuang}
\email{bzhuang@hmc.edu}
\affiliation{Harvey Mudd College}
\date{\today}


%%%%%%%%%%%
% abstract
%%%%%%%%%%%

\begin{abstract}

This notes explore the units and the magnitudes of numerical quantities used in the polarizability calculation.
	
\end{abstract}

\maketitle

\section{Water}

Let us first take a look at water's polarizability and its corresponding index of refraction. This would help us to get a sense of the magnitudes of the quantities in the different units.

\subsection{CGS unit}

 Water polarizability is $\alpha = 1.45$ \AA$^3$ $= 1.45 \times 10^{-24}\ \text{cm}^3$ expressed in the cgs unit.

 To calculate the index of refraction, we can use the Lorentz-Lorenz relation in cgs unit given by
\begin{equation}
\alpha = \frac{3}{4\pi n} \left( \frac{\eta^2 - 1 }{\eta^2 +2} \right)
\end{equation}
where $\eta$ is the index of refraction and $n$ is the number density of the water molecule. Rearranging the equation gives
\begin{eqnarray}
\frac{4\pi n\alpha}{3} &=&   \left( \frac{\eta^2 - 1 }{\eta^2 +2} \right) \nonumber\\
\gamma &=&   \left( \frac{\eta^2 - 1 }{\eta^2 +2} \right) \nonumber\\
\gamma\eta^2 +2\gamma &=&  \eta^2 - 1\nonumber\\
2\gamma + 1 &=&  (1-\gamma) \eta^2\nonumber\\ 
\eta &=&  \sqrt{\frac{1 + 2\gamma}{1-\gamma}}
\label{refracIndex}
\end{eqnarray}
where we have set $\gamma = \frac{4\pi n\alpha}{3}$ in the second equality. 


Water's density at room temperature is 0.997 g/cm$^3$ and its molar mass is 18.01528 g/mol. The number density can be worked out as
\begin{eqnarray}
n &=&  \frac{(0.997\ \text{g/cm}^3) \left( \frac{1\ \text{cm}^3}{10^{24}\ \text{\AA}^3} \right)}{(18.01528\ \text{g/mol}) \left( \frac{1\ \text{mol}}{6.02\ \times 10^{23}\ \text{molecules} }\right)}\nonumber\\
&=& 0.03332\ \text{\AA}^{-3}
\label{numberDenisty}
\end{eqnarray}


Substituting the various values into Eq.~\eqref{refracIndex}, we have the following value for refractive index of water
\begin{eqnarray}
\gamma &=& \frac{4\pi n\alpha}{3} = \frac{4\pi (0.03332\ \text{\AA}^{-3})( 1.45\ \text{\AA}^3)}{3} =0.2024 \nonumber\\
\eta &=&  \sqrt{\frac{1 + 2\times 0.2024 }{1-0.2024}} = 1.327
\end{eqnarray}
which is consistent with water's measured refractive index at 1.33 at 20$^\circ$C.

Note that the optical dielectric constant (the part of the dielectric constant due to polarizability) is given by $\varepsilon = n^2$.   


\section{LAMMPS Calculation Unit}

We run the simulations in Lennard-Jones Units in LAMMPS. The Lennard-Jones Units is defined by Lennard-Jones length $\sigma$ and Lennard-Jones energy $\epsilon$.

There is no other length unit in the LAMMPS calculation. So the polarizability must have been in units of $\sigma^3$. 

Currently, we have $n = \frac{900}{11^3 \sigma^3}$ and $\alpha = 1.5 \sigma^3$. That would make $\gamma>1$, possibly this is too large. (Is it because of this that our Drude electrons are flying away?)


\section{Reasonable value to use for water?}

\subsection{Length and density}

Let's try to come up with a set of parameters to use for water based on the SWM4-NDP model of water. In that model, $\sigma = 3.18395\ \text{\AA}$. Therefore, the polarizability is $\alpha = 1.45\ \text{\AA}^3 = 0.0449 \sigma^3$. Water's density is $0.03332\ \text{\AA}^{-3}$, that is a density of $n = 1.0755 \sigma^{-3}$ in Lennard Jones Units.


\subsection{Energy and temperature}

In the SWM4-NDP model, we also have the energy unit $\epsilon = 0.21094\ \text{kcal/mol} = 1.466 \times 10^{-21}\ \text{J/molecule}$, so the temperature unit is $\epsilon/k_B = 106.2\ \text{K}$. That is to say, if we want to simulate a temperature at 300 K, it is 2.823 $\epsilon/k_B$ as temperature in Lennard-Jones units. Similarly, for the Drude oscillator at 1 K, it is 0.009413  $\epsilon/k_B$ as temperature in Lennard-Jones units.
 

\subsection{Time}

The dimension of energy is 
\begin{equation}
[\text{energy}] = [\text{mass}][\text{length}]^2[\text{time}]^{-2}
\end{equation}
so the unit of time (let us denote it by $\tau$) in LAMMPS is given by
\begin{equation}
\tau = \left(\frac{m \sigma^2}{\epsilon}\right)^\frac{1}{2} 
\label{tau}
\end{equation}
where the mass $m$ is the mass of the particle.

If we convert this to the SI unit, this is
\begin{eqnarray}
\tau &=&  \left(\frac{m \sigma^2}{\epsilon}\right)^\frac{1}{2} \nonumber\\
&=& \left(\frac{(18 \times 1.66 \times 10^{-27}\ \text{kg/molecule} ) (3.18395\ \times 10^{-10} \text{m})^2}{1.466 \times 10^{-21}\ \text{J/molecule} }\right)^\frac{1}{2} \nonumber\\
&=& 1.437 \times 10^{-12}\ \text{s} 
\end{eqnarray}

In the Langevin integrator, the damping coefficient is approximately $1/(10\ \text{ps}^{-1})$. We can convert that to the LJ unit as follows:
\begin{equation}
\frac{1}{(10\ \text{ps}^{-1})} = \frac{1}{(10\ \text{ps}^{-1})} \frac{1\ \text{s}}{10^{12} \text{ps}}  \frac{\tau}{ 1.437 \times 10^{-12}\ \text{s}}
\end{equation}
%



%%%%%%%%%%%%%%%%%%%%%%%%%%%%


\bibliography{$HOME/Research_Literature/Research_General_Literature/General_lit_bib.bib}

\end{document}



